% По умолчанию используется шрифт 14 размера. Если нужен 12-й шрифт, уберите опцию [14pt]
\documentclass[14pt
  , russian
  %, xcolor={svgnames}
  ]{matmex-diploma-custom}
\usepackage[table]{xcolor}
\usepackage{graphicx}
\usepackage{tabularx}
\newcolumntype{Y}{>{\centering\arraybackslash}X}
\usepackage{amsmath}
\usepackage{amsthm}
\usepackage{amsfonts}
\usepackage{amssymb}
\usepackage{mathtools}
\usepackage{thmtools}
\usepackage{thm-restate}
\usepackage{tikz}
\usepackage{wrapfig}
% \usepackage[kpsewhich,newfloat]{minted}
% \usemintedstyle{vs}
\usepackage[inline]{enumitem}
\usepackage{subcaption}
\usepackage{caption}
\usepackage[nocompress]{cite}
\usepackage{makecell}
% \setitemize{noitemsep,topsep=0pt,parsep=0pt,partopsep=0pt}
% \setenumerate{noitemsep,topsep=0pt,parsep=0pt,partopsep=0pt}


\graphicspath{ {resources/} }

% 
% % \documentclass 
% %   [ a4paper        % (Predefined, but who knows...)
% %   , draft,         % Show bad things.
% %   , 12pt           % Font size.
% %   , pagesize,      % Writes the paper size at special areas in DVI or
% %                    % PDF file. Recommended for use.
% %   , parskip=half   % Paragraphs: noindent + gap.
% %   , numbers=enddot % Pointed numbers.
% %   , BCOR=5mm       % Binding size correction.
% %   , submission
% %   , copyright
% %   , creativecommons 
% %   ]{eptcs}
% % \providecommand{\event}{ML 2018}  % Name of the event you are submitting to
% % \usepackage{breakurl}             % Not needed if you use pdflatex only.
% 
% \usepackage{underscore}           % Only needed if you use pdflatex.
% 
% \usepackage{booktabs}
% \usepackage{amssymb}
% \usepackage{amsmath}
% \usepackage{mathrsfs}
% \usepackage{mathtools}
% \usepackage{multirow}
% \usepackage{indentfirst}
% \usepackage{verbatim}
% \usepackage{amsmath, amssymb}
% \usepackage{graphicx}
% \usepackage{xcolor}
% \usepackage{url}
% \usepackage{stmaryrd}
% \usepackage{xspace}
% \usepackage{comment}
% \usepackage{wrapfig}
% \usepackage[caption=false]{subfig}
% \usepackage{placeins}
% \usepackage{tabularx}
% \usepackage{ragged2e}
% \usepackage{soul}
\usepackage{csquotes}
% \usepackage{inconsolata}
% 
% \usepackage{polyglossia}   % Babel replacement for XeTeX
%   \setdefaultlanguage[spelling=modern]{russian}
%   \setotherlanguage{english}
% \usepackage{fontspec}    % Provides an automatic and unified interface 
%                          % for loading fonts.
% \usepackage{xunicode}    % Generate Unicode chars from accented glyphs.
% \usepackage{xltxtra}     % "Extras" for LaTeX users of XeTeX.
% \usepackage{xecyr}       % Help with Russian.
% 
% %% Fonts
% \defaultfontfeatures{Mapping=tex-text}
% \setmainfont{CMU Serif}
% \setsansfont{CMU Sans Serif}
% \setmonofont{CMU Typewriter Text}

\usepackage[final]{listings}

\lstdefinelanguage{ocaml}{
keywords={@type, function, fun, let, in, match, with, when, class, type,
nonrec, object, method, of, rec, repeat, until, while, not, do, done, as, val, inherit, and,
new, module, sig, deriving, datatype, struct, if, then, else, open, private, virtual, include, success, failure,
lazy, assert, true, false, end},
sensitive=true,
commentstyle=\small\itshape\ttfamily,
keywordstyle=\ttfamily\bfseries, %\underbar,
identifierstyle=\ttfamily,
basewidth={0.5em,0.5em},
columns=fixed,
fontadjust=true,
literate={->}{{$\to$}}3 {===}{{$\equiv$}}1 {=/=}{{$\not\equiv$}}1 {|>}{{$\triangleright$}}3 {\\/}{{$\vee$}}2 {/\\}{{$\wedge$}}2 {>=}{{$\ge$}}1 {<=}{{$\le$}} 1,
morecomment=[s]{(*}{*)}
}

\lstset{
mathescape=true,
%basicstyle=\small,
identifierstyle=\ttfamily,
keywordstyle=\bfseries,
commentstyle=\scriptsize\rmfamily,
basewidth={0.5em,0.5em},
fontadjust=true,
language=ocaml
}
 
\newcommand{\cd}[1]{\texttt{#1}}
\newcommand{\inbr}[1]{\left<#1\right>}


\newcolumntype{L}[1]{>{\raggedright\let\newline\\\arraybackslash\hspace{0pt}}m{#1}}
\newcolumntype{C}[1]{>{\centering\let\newline\\\arraybackslash\hspace{0pt}}m{#1}}
\newcolumntype{R}[1]{>{\raggedleft\let\newline\\\arraybackslash\hspace{0pt}}m{#1}}



\usepackage{soul}
\usepackage[normalem]{ulem}
%\sout{Hello World}

% перевод заголовков в листингах
\renewcommand\lstlistingname{Листинг}
\renewcommand\lstlistlistingname{Листинги}

\usepackage{listings}
\usepackage{tikz}
\usetikzlibrary{decorations.pathreplacing,calc,shapes,positioning,tikzmark}

\newcounter{tmkcount}

\tikzset{
  use tikzmark/.style={
    remember picture,
    overlay,
    execute at end picture={
      \stepcounter{tmkcount}
    },
  }%,
  %tikzmark suffix={-\thetmkcount}
}

\usepackage{caption}
\usepackage{listings}

\DeclareCaptionFont{white}{ \color{white} }
\DeclareCaptionFormat{listing}{
    \parbox{\textwidth}{\hspace{15pt}#1#2#3}
}
\captionsetup[lstlisting]{ format=listing
  %, labelfont=white, textfont=white
  , singlelinecheck=false, margin=0pt, font={bf}
}

\begin{document}
%% Если что-то забыли, при компиляции будут ошибки Undefined control sequence \my@title@<что забыли>@ru
%% Если англоязычная титульная страница не нужна, то ее можно просто удалить.
\filltitle{ru}{
    %% Актуально только для курсовых/практик. ВКР защищаются не на кафедре а в ГЭК по направлению, 
    %%   и к моменту защиты вы будете уже не в группе.
    chair              = {Кафедра системного программирования},
    group              = {17Б.11-мм},
    %% Макрос filltitle ненавидит пустые строки, поэтому обязателен хотя бы символ комментария на строке
    %% Актуально всем.
    title              = {Поиск энергетически неэффективного кода в приложениях для Android},
    % 
    %% Здесь указывается тип работы. Возможные значения:
    %%   coursework - отчёт по курсовой работе;
    %%   practice - отчёт по учебной практике;
    %%   prediploma - отчёт по преддипломной практике;
    %%   master - ВКР магистра;
    %%   bachelor - ВКР бакалавра.
    type               = {practice},
    author             = {БОГДАНОВ Егор Дмитриевич},
    % 
    %% Актуально только для ВКР. Указывается код и название направления подготовки. Типичные примеры:
    %%   02.03.03 <<Математическое обеспечение и администрирование информационных систем>>
    %%   02.04.03 <<Математическое обеспечение и администрирование информационных систем>>
    %%   09.03.04 <<Программная инженерия>>
    %%   09.04.04 <<Программная инженерия>>
    %% Те, что с 03 в середине --- бакалавриат, с 04 --- магистратура.
    specialty          = {09.03.04 <<Программная инженерия>>},
    % 
    %% Актуально только для ВКР. Указывается шифр и название образовательной программы. Типичные примеры:
    %%   СВ.5006.2017 <<Математическое обеспечение и администрирование информационных систем>>
    %%   СВ.5162.2020 <<Технологии программирования>>
    %%   СВ.5080.2017 <<Программная инженерия>>
    %%   ВМ.5665.2019 <<Математическое обеспечение и администрирование информационных систем>>
    %%   ВМ.5666.2019 <<Программная инженерия>>
    %% Шифр и название программы можно посмотреть в учебном плане, по которому вы учитесь. 
    %% СВ.* --- бакалавриат, ВМ.* --- магистратура. В конце --- год поступления (не обязательно ваш, если вы были в академе/вылетали).
    programme          = {СВ.5080.2017 <<Программная инженерия>>},
    % 
    %% Актуально только для ВКР, только для матобеса и только 2017-2018 годов поступления. Указывается профиль подготовки, на котором вы учитесь.
    %% Названия профилей можно найти в учебном плане в списке дисциплин по выбору. На каком именно вы, вам должны были сказать после второго курса (можно уточнить в студотделе).
    %% Вот возможные вариканты:
    %%   Математические основы информатики
    %%   Информационные системы и базы данных
    %%   Параллельное программирование
    %%   Системное программирование
    %%   Технология программирования
    %%   Администрирование информационных систем
    %%   Реинжиниринг программного обеспечения
    % profile            = {Системное программирование},
    % 
    %% Актуально всем.
    %supervisorPosition = {проф. каф. СП, д.ф.-м.н., проф.}, % Терехов А.Н.
    supervisorPosition = {к.ф.-м.н., доцент кафедры системного программирования,}, % Григорьев С.В.   
    supervisor         = {Д.В. Луцив},  
    % 
    %% Актуально только для практик и курсовых. Если консультанта нет, закомментировать или удалить вовсе.
    consultantPosition = {ст. преп. кафедры системного программирования},
    consultant         = {С.Ю. Сартасов},
    %
    %% Актуально только для ВКР.
    reviewerPosition   = {должность ООО <<Место работы>> степень},
    reviewer           = {Р.Р. Рецензент},
}

% \filltitle{en}{
%     chair              = {Advisor's chair},
%     group              = {ХХB.ХХ-mm},
%     title              = {Template for SPbU qualification works},
%     type               = {practice},
%     author             = {FirstName Surname},
%     % 
%     %% Possible choices:
%     %%   02.03.03 <<Software and Administration of Information Systems>>
%     %%   02.04.03 <<Software and Administration of Information Systems>>
%     %%   09.03.04 <<Software Engineering>>
%     %%   09.04.04 <<Software Engineering>>
%     %% Те, что с 03 в середине --- бакалавриат, с 04 --- магистратура.
%     specialty          = {02.03.03 ``Software and Administration of Information Systems''},
%     % 
%     %% Possible choices:
%     %%   СВ.5006.2017 <<Software and Administration of Information Systems>>
%     %%   СВ.5162.2020 <<Programming Technologies>>
%     %%   СВ.5080.2017 <<Software Engineering>>
%     %%   ВМ.5665.2019 <<Software and Administration of Information Systems>>
%     %%   ВМ.5666.2019 <<Software Engineering>>
%     programme          = {СВ.5006.2017 ``Software and Administration of Information Systems''},
%     % 
%     %% Possible choices:
%     %%   Mathematical Foundations of Informatics
%     %%   Information Systems and Databases
%     %%   Parallel Programming
%     %%   System Programming
%     %%   Programming Technology
%     %%   Information Systems Administration
%     %%   Software Reengineering
%     % profile            = {Software Engineering},
%     % 
%     %% Note that common title translations are:
%     %%   кандидат наук --- C.Sc. (NOT Ph.D.)
%     %%   доктор ... наук --- Sc.D.
%     %%   доцент --- docent (NOT assistant/associate prof.)
%     %%   профессор --- prof.
%     supervisorPosition = {Sc.D, prof.},
%     supervisor         = {S.S. Supervisor},
%     % 
%     consultantPosition = {position at ``Company'', degree if present},
%     consultant         = {C.C. Consultant},
%     %
%     reviewerPosition   = {position at ``Company'', degree if present},
%     reviewer           = {R.R. Reviewer},
% }
\maketitle
\setcounter{tocdepth}{2}
\tableofcontents

% \begin{abstract}
%   В курсаче не нужен
% \end{abstract}

%\section{Введение}
%Мобильные устройства пользовались огромным успехом за последние пять лет, например, в последние 5 лет средние продажи смартфонов составляют более 1,5 млн. в год~\cite{globalsales}, и более 70\% из этих устройств используют операционную систему Android~\cite{worldwide}. По мере увеличения количества устройств количество приложений (или приложений) также быстро росло в последние годы. Следовательно, количество мобильных разработчиков также увеличивается. Приложения в основном пишутся с использованием популярных объектно-ориентиро\-ванных (или объектно-ориентированных) языков программирования, таких как Java, Objective-C, Swift или C\#. 

 Тем не менее, мобильная разработка не так похожа на традиционную разработку программного обеспечения~\cite{wasserman2010software}, и разработчики должны учитывать мобильные особенности. Кроме того, спрос пользователей продолжает расти и вынуждает мобильных разработчиков как можно быстрее добавлять новые функции и поддерживать свои приложения. К сожалению, это давление заставляет разработчиков применять плохие методы реализации, также известные как запахи кода~\cite{fowler2018refactoring}. 
 
 Запахи кода могут привести к утечке ресурсов ЦП, памяти, батареи и т.д.~\cite{cscatalogue}. Утечки могут ухудшить качество приложения с точки зрения стабильности, удобства для пользователя, удобства обслуживания и т.д. Также важно отметить, что более 18\% приложений Android имеют запахи кода~\cite{liu2014characterizing}.
 
 В связи с потенциальным влияением на энергопотребление устройства, а следовательно и времени его работы, необходимо выяснить, какие энергетические запахи кода были изучены. На основе полученных знаний также необходимо разработать инструмент автоматического исправления энергетических запахов кода, который может помочь разработчикам мобильных приложений в создании более энергоэффективного кода.


% Больше инфы можно поискать в слайдах Кознова%~\cite{koznov}.

% Тут  4 части(абзаца) максимум на 2 страницы
% \begin{enumerate}
% \item Background, known information
% \item Knowledge gap, unknown information
% \item  Hypothesis, question, purpose statement 
% \item Approach, plan of attack, proposed solution
% \begin{itemize}
% \item Последний абзац должен читаться и быть понятным в отрыве от остальных трёх
% \end{itemize}
% \end{enumerate}

%\section{Постановка задачи}
%\label{sec:task}
%% !TeX spellcheck = ru_RU
Целью работы является создание или расширение существующего инструмента для нахождения и исправления энергетических запахов кода на этапе написания кода. 
 
Для её выполнения были поставлены следующие задачи:

\begin{itemize}
    \item Провести обзор работ, исследующих энергетические запахи кода;
    \item Провести обзор инструментов, которые находят и автоматически исправляют эенергетические запахи кода;
    \item Выбрать инструмент для дальнейшего расширения;
    \item Выбрать энергетические запахи кода, поддержка которых будет внедряться;
    \item Реализовать поддержку выбранных энергетических запахов кода;
    \item Произвести апробацию полученного инструмента.     
\end{itemize}


\section{Обзор}
\label{sec:relatedworks}


\subsection{Аналогичные исследования}


Для поиска аналогичных исследований, связанных с энергетическими запахами кода Android, использовался сервис “Google Scholar” с поисковыми запросами по ключевым словам: \texttt{android}, \texttt{energy code smell}, \texttt{energy anti-patterns}, \texttt{refactoring}. Из полученных результатов были отобраны работы, темы которых были схожи с исследованием запахов кода или разработкой инструмента для обнаружения или исправления запахов кода. На втором этапе были изучены другие работы авторов отобранных статей, а также работы, в которых отобранные на первом шаге статьи цитировались. Таким образом был найден список отобранных статей, посвященных энергоэффективной Android разработке, среди которых были найдены новые статьи, связанные с темой данной работы. Также в процессе поиска была найдена работа, текст которой отсутствовал в открытом доступе, поэтому мы связались с одним из авторов, и он выслал текст статьи.

Gottschalk et al. \cite{gottschalk2016refactorings} определяют и исследуют такие запахи кода, как \emph{Data Transfer}, \emph{Backlight}, \emph{Statement Change}, \emph{Third-Party Advertising}, \emph{Binding Resources Too Early}. Для выявления этих запахов авторы использовали графовое представление исходного кода и взаимодействовали с этим графом с помощью библиотеки JGraLab\footnote{http://jgralab.github.io/}. В результате работы было выявлено, что все исследуемые запахи кода негативно влияют на энергопотребление приложения. Также исследователи составили список теоретически энергозатратных запахов кода для изучения в будущих работах.

Cruz et al. в своих работах исследовали влияние на энергопотребление наиболее часто встречаемых ошибок, которые распознает Android Lint\footnote{http://tools.android.com/tips/lint}: \emph{DrawAllocation}, \emph{WakeLock}, \emph{Recycle}, \emph{ObsoleteLayoutParam}, \emph{ViewHolder}, \emph{Overdraw}, \emph{UnusedResources}, \emph{UselessParent}.
В статье Performance-based guidelines \cite{cruz2017performance} авторы измеряли влияние этих запахов на энергопотребление и пришли к выводу, что исправление запахов кода \emph{DrawAllocation}, \emph{WakeLock}, \emph{Recycle}, \emph{ObsoleteLayoutParam} и \emph{ViewHolder} положительно влияет на энергозатраты приложения. 
В следующей статье \cite{cruz2018using} исследователи представили инструмент для автоматического рефакторинга энергетических антипаттернов, выделенных в предыдущей статье. За основу был взят инструмент AutoRefactor, который является плагином для IDE Eclipse. Авторы добавили в него поддержку автоматического исправления выбранных антипаттернов. 

Reimann et al. \cite{reimann2014tool} представили каталог\footnote{https://martinbrylski.github.io/android\_smells/} специфических для Android антипаттернов с предложениями по их исправлению. Среди этих антипаттернов исследователи выделили паттерны, теоретически влияющих на энергопотребление. Данный каталог основан на работах Фаулера и Брауна с учетом особенности Android разработки. 

Carette et al. \cite{carette2017investigating} исследовали антипаттерны, выбранные на основе результатов своих предыдущих работ и советов по разработке в официальной документации Android: \emph{HashMap Usage}, \emph{Internal Getter/Setter} и \emph{Member Ignoring Method}. В этой работе исследователи представили инструмент для автоматического рефакторинга HotPepper, поддерживающего исследуемые запахи кода. Этот инструмент состоит из двух частей. Первая часть основана на инструменте Paprika (инструмент для обнаружения специфических для Android антипаттернов, который анализирует скомпилированный код приложения и выдаёт список найденных запахов кода). Вторая часть исправляет найденные ошибки в исходном коде программы. Для этого исходный код представляется в виде абстрактного синтаксического дерева, которое изменяется с помощью библиотеки SPOON. В результате работы было выявлено, что исправление каждого запаха кода положительно влияет на энергопотребление приложений.

Palomba et al. \cite{palomba2019impact} исследовали 9 запахов кода, выбранных из каталога Рейманна: \emph{Data Transmission Without Compression}, \emph{Durable Wakelock}, \emph{Inefficient Data Structure}, \emph{Inefficient SQL Query}, \emph{Inefficient Data Format And Parser}, \emph{Internal Setter}, \emph{Leaking Thread}, \emph{Member Ignoring Method и Slow Loop}. Для поиска этих антипаттернов их предыдущая разработка aDoctor \cite{palomba2017lightweight}. В результате проведенных экспериментов антипаттерны \emph{Internal Setter}, \emph{Leaking Thread}, \emph{Member Ignoring Method} и \emph{Slow Loop} оказались негативно влияющими на энергопотребление.

Morales et al. \cite{morales2016anti} изучали запахи кода, полученные на основе работ Фаулера \cite{fowler2018refactoring}, Готтшалк \cite{gottschalk2016refactorings} и советов по разработке в официальной документации Android: \emph{Blob}, \emph{Lazy Class}, \emph{Long-parameter list}, \emph{Refused Bequest}, \emph{Speculative Generality}, \emph{Binding Resources too early}, \emph{HashMap usage} и \emph{Private getters and setters}. Для обнаружения этих антипаттернов исследователи использовали инструмент собственной разработки ReCon \cite{morales2017use}, добавив в него поддержку вышеописанных запахов кода. В качестве тестируемого кода были взято несколько проектов из репозитория FDroid\footnote{https://f-droid.org/}, на которых авторы использовали инструмент и вручную исправляли найденные антипаттерны. В результате экспериментов были выделены энергозатратные антипаттерны: \emph{Blob}, \emph{Lazy Class},  \emph{Refused Bequest}, \emph{Binding Resources too early}, \emph{HashMap usage} и \emph{Private getters and setters}.

Fatima et al. \cite{fatima2020tool} провели исследование и сделали обзор существующих инструментов поддержки Android разработки. Найденные инструменты были разделены на 3 категории: профилировщики, детекторы специфических для Android запахов кода и оптимизаторы. В статье был выделен список антипаттернов, которые покрываются найденными инструментами, а также список используемых для их обнаружения техник и типы входных и выходных данных каждого инструмента (например, исходный код приложения или APK). Наиболее типичной техникой обнаружения был статический анализ кода с использованием предопределённого набора правил и запахов кода. В своей следующей работе \cite{fatima2020detection} исследователи представили собственный инструмент обнаружения и исправления запахов кода, основанный на выделенных в предыдущей работе идеях. Новый инструмент является расширением существующего инструмента анализа кода Android Lint. Авторы взяли его за основу, потому что Android Lint имеет интеграцию в официальную среду Android разработки Android Studio, а также имеет API, который позволяет добавлять собственные правила анализа кода. Таким образом исследователи реализовали поддержку 9 новых специфических для Android запахов кода и улучшили поддержку 3 существующих запахов кода.

\begin{table}[]
\begin{tabular}{|l|l|l|l|l|l|}
\hline
\begin{tabular}[c]{@{}l@{}}Подтверждённые \\ антипаттерны\end{tabular} & \begin{tabular}[c]{@{}l@{}}Gottschalk\\ et al.\end{tabular} & \begin{tabular}[c]{@{}l@{}}Cruz\\ et al.\end{tabular} & \begin{tabular}[c]{@{}l@{}}Carette \\ et al.\end{tabular} & \begin{tabular}[c]{@{}l@{}}Palomba \\ et al.\end{tabular} & \begin{tabular}[c]{@{}l@{}}Morales \\ et al.\end{tabular} \\ \hline
Data Transfer                                                          & \checkmark                                    &                                                       &                                                           &                                                           &                                                           \\ \hline
Backlight                                                              & \checkmark                                    &                                                       &                                                           &                                                           &                                                           \\ \hline
Statement Change                                                       & \checkmark                                    &                                                       &                                                           &                                                           &                                                           \\ \hline
\begin{tabular}[c]{@{}l@{}}Third-Party \\ Advertising\end{tabular}     & \checkmark                                    &                                                       &                                                           &                                                           &                                                           \\ \hline
\begin{tabular}[c]{@{}l@{}}Binding Resource \\ Too Early\end{tabular}  & \checkmark                                    &                                                       &                                                           &                                                           & \checkmark                                  \\ \hline
DrawAllocation                                                         &                                                             & \checkmark                              &                                                           &                                                           &                                                           \\ \hline
WakeLock                                                               &                                                             & \checkmark                              &                                                           &                                                           &                                                           \\ \hline
Recycle                                                                &                                                             & \checkmark                              &                                                           &                                                           &                                                           \\ \hline
ObsoleteLayoutParam                                                    &                                                             & \checkmark                              &                                                           &                                                           &                                                           \\ \hline
ViewHolder                                                             &                                                             & \checkmark                              &                                                           &                                                           &                                                           \\ \hline
HashMap Usage                                                          &                                                             &                                                       & \checkmark                                  &                                                           & \checkmark                                  \\ \hline
Internal Getter/Setter                                                 &                                                             &                                                       & \checkmark                                  & \checkmark                                  & \checkmark                                  \\ \hline
\begin{tabular}[c]{@{}l@{}}Member \\ Ignoring \\ Method\end{tabular}   &                                                             &                                                       & \checkmark                                  & \checkmark                                  &                                                           \\ \hline
Leaking Thread                                                         &                                                             &                                                       &                                                           & \checkmark                                  &                                                           \\ \hline
Slow Loop                                                              &                                                             &                                                       &                                                           & \checkmark                                  &                                                           \\ \hline
Blob                                                                   &                                                             &                                                       &                                                           &                                                           & \checkmark                                  \\ \hline
Lazy Class                                                             &                                                             &                                                       &                                                           &                                                           & \checkmark                                  \\ \hline
Refused Bequest                                                        &                                                             &                                                       &                                                           &                                                           & \checkmark                                  \\ \hline
\end{tabular}
\caption{Подтверждённые запахи кода и работы, в которых они исследовались.}
\end{table}



% \section{Background (опционально)}
% TODO


% \section{Метод}
% Реализация в широком смысле: что таки было сделано. Скорее всего самый большой раздел.

% \emph{Крайне желательно} к Новому году иметь что-то, что сюда можно написать.


% \subsection{Основные грабли}
% Здесь мы будем собирать основные ошибки, которые случаются при написании текстов.


% \begin{lstlisting}[caption=Название, language=Caml, frame=single]
% let x = 5 in x+1
% \end{lstlisting}

% \subsection{Выделения куска листинга с помощью tikz}
% \url{https://tex.stackexchange.com/questions/284311}

% \begin{lstlisting}[escapechar=!,basicstyle=\ttfamily]
% #include <stdio.h>
% #include <math.h>

% int main () {
%   double c=-1;
%   double z=0;
%   int i;

%   printf (``For c = %lf:\n'', c );
%   for ( i=0; i<10; i++ ) {
%     printf ( !\tikzmark{a}!"z %d = %lf\n"!\tikzmark{b}!, i, z );
%     z = pow(z,2) + c;
%   }
% }
% \end{lstlisting}

% \begin{tikzpicture}[use tikzmark]
% \draw[fill=gray,opacity=0.1]
%   ([shift={(-3pt,2ex)}]pic cs:a) 
%     rectangle 
%   ([shift={(3pt,-0.65ex)}]pic cs:b);
% \end{tikzpicture}


% \section{Эксперимент}
% Как мы проверяем, что  всё удачно получилось

% \subsection{Условия эксперимента}
% Железо (если актуально); входные данные, на которых проверяем наш подход; почему мы выбрали именно эти тесты

% \subsection{Исследовательские вопросы (Research questions)}
% Надо сформулировать то, чего мы хотели бы добиться работой (2 штуки будет хорошо):

% \begin{itemize}
% \item Хотим алгоритм, который лучше вот таких-то остальных
% \item Если в подходе можно включать/выключать составляющие, то насколько существенно каждая составляющая влияет на улучшения
% \item Если у нас строится приближение каких-то штук, то на сколько точными будут эти приближения
% \item и т.п.
% \end{itemize}

% \subsection{Метрики}

% Как мы сравниваем, что результаты двух подходов лучше или хуже
% \begin{itemize}
% \item Производительность
% \item Строчки кода
% \item Как часто алгоритм "угадывает" правильную классификацию входа
% \end{itemize}

% Иногда метрики вырожденные (да/нет), это не очень хорошо, но если в области исследований так принято, то ладно.

% \subsection{Результаты}
% Результаты понятно что такое. Тут всякие таблицы и графики

% В этом разделе надо также коснуться Research Questions.

% \subsubsection{RQ1} Пояснения
% \subsubsection{RQ2} Пояснения

% \subsection{Обсуждение результатов}

% Чуть более неформальное обсуждение, то, что сделано. Например, почему метод работает лучше остальных? Или, что делать со случаями, когда метод классифицирует вход некорректно.

% \section{Применение того, что сделано на практике (опциональный)}

% Если применение в лоб не работает, потому что всё изложено чуть более сжато и теоретично, надо рассказать тонкости и правильный метод применения результатов. 

% \section{Угрозы нарушения корректности (опциональный)}

% Если основная заслуга метода, это то, что он дает лучшие цифры, то стоит сказать, где мы могли облажаться, когда проводили численные замеры. 

% \section{Заключение (обязательно в осеннем семестре)}

% Кратко, что было сделано. Также здесь стоит писать задачи на будущее.

% \textbf{Для практик/ВКР.} Также важно сделать список результатов, который будет один к одному соответствовать задачам из раздела~\ref{sec:task}.

% \begin{itemize}
% \item Результат к задаче 1 
% \item Результат к задаче 2
% \item и т.д.
% \end{itemize}
% \noindent Для промежуточных отчетов сюда важно записать какие задачи уже были сделаны за осенний семестр, а какие только планируется сделать.

% Также сюда можно писать планы развития работы в будущем, или, если их много, выделять под это отдельную главу.


% \nocite{*}
\setmonofont[Mapping=tex-text]{CMU Typewriter Text}
\bibliographystyle{ugost2008ls}
\bibliography{vkr}
\end{document}
