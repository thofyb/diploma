\label{sec:relatedworks}


\subsection{Аналогичные исследования}


Для поиска аналогичных исследований, связанных с энергетическими запахами кода Android, использовался сервис “Google Scholar” с поисковыми запросами по ключевым словам: \texttt{android}, \texttt{energy code smell}, \texttt{energy anti-patterns}, \texttt{refactoring}. Из полученных результатов были отобраны работы, темы которых были схожи с исследованием запахов кода или разработкой инструмента для обнаружения или исправления запахов кода. На втором этапе были изучены другие работы авторов отобранных статей, а также работы, в которых отобранные на первом шаге статьи цитировались. Таким образом был найден список отобранных статей, посвященных энергоэффективной Android разработке, среди которых были найдены новые статьи, связанные с темой данной работы. Также в процессе поиска была найдена работа, текст которой отсутствовал в открытом доступе, поэтому мы связались с одним из авторов, и он выслал текст статьи.

Gottschalk et al. \cite{gottschalk2016refactorings} определяют и исследуют такие запахи кода, как \emph{Data Transfer}, \emph{Backlight}, \emph{Statement Change}, \emph{Third-Party Advertising}, \emph{Binding Resources Too Early}. Для выявления этих запахов авторы использовали графовое представление исходного кода и взаимодействовали с этим графом с помощью библиотеки JGraLab\footnote{http://jgralab.github.io/}. В результате работы было выявлено, что все исследуемые запахи кода негативно влияют на энергопотребление приложения. Также исследователи составили список теоретически энергозатратных запахов кода для изучения в будущих работах.

Cruz et al. в своих работах исследовали влияние на энергопотребление наиболее часто встречаемых ошибок, которые распознает Android Lint\footnote{http://tools.android.com/tips/lint}: \emph{DrawAllocation}, \emph{WakeLock}, \emph{Recycle}, \emph{ObsoleteLayoutParam}, \emph{ViewHolder}, \emph{Overdraw}, \emph{UnusedResources}, \emph{UselessParent}.
В статье Performance-based guidelines \cite{cruz2017performance} авторы измеряли влияние этих запахов на энергопотребление и пришли к выводу, что исправление запахов кода \emph{DrawAllocation}, \emph{WakeLock}, \emph{Recycle}, \emph{ObsoleteLayoutParam} и \emph{ViewHolder} положительно влияет на энергозатраты приложения. 
В следующей статье \cite{cruz2018using} исследователи представили инструмент для автоматического рефакторинга энергетических антипаттернов, выделенных в предыдущей статье. За основу был взят инструмент AutoRefactor, который является плагином для IDE Eclipse. Авторы добавили в него поддержку автоматического исправления выбранных антипаттернов. 

Reimann et al. \cite{reimann2014tool} представили каталог\footnote{https://martinbrylski.github.io/android\_smells/} специфических для Android антипаттернов с предложениями по их исправлению. Среди этих антипаттернов исследователи выделили паттерны, теоретически влияющих на энергопотребление. Данный каталог основан на работах Фаулера и Брауна с учетом особенности Android разработки. 

Carette et al. \cite{carette2017investigating} исследовали антипаттерны, выбранные на основе результатов своих предыдущих работ и советов по разработке в официальной документации Android: \emph{HashMap Usage}, \emph{Internal Getter/Setter} и \emph{Member Ignoring Method}. В этой работе исследователи представили инструмент для автоматического рефакторинга HotPepper, поддерживающего исследуемые запахи кода. Этот инструмент состоит из двух частей. Первая часть основана на инструменте Paprika (инструмент для обнаружения специфических для Android антипаттернов, который анализирует скомпилированный код приложения и выдаёт список найденных запахов кода). Вторая часть исправляет найденные ошибки в исходном коде программы. Для этого исходный код представляется в виде абстрактного синтаксического дерева, которое изменяется с помощью библиотеки SPOON. В результате работы было выявлено, что исправление каждого запаха кода положительно влияет на энергопотребление приложений.

Palomba et al. \cite{palomba2019impact} исследовали 9 запахов кода, выбранных из каталога Рейманна: \emph{Data Transmission Without Compression}, \emph{Durable Wakelock}, \emph{Inefficient Data Structure}, \emph{Inefficient SQL Query}, \emph{Inefficient Data Format And Parser}, \emph{Internal Setter}, \emph{Leaking Thread}, \emph{Member Ignoring Method и Slow Loop}. Для поиска этих антипаттернов их предыдущая разработка aDoctor \cite{palomba2017lightweight}. В результате проведенных экспериментов антипаттерны \emph{Internal Setter}, \emph{Leaking Thread}, \emph{Member Ignoring Method} и \emph{Slow Loop} оказались негативно влияющими на энергопотребление.

Morales et al. \cite{morales2016anti} изучали запахи кода, полученные на основе работ Фаулера \cite{fowler2018refactoring}, Готтшалк \cite{gottschalk2016refactorings} и советов по разработке в официальной документации Android: \emph{Blob}, \emph{Lazy Class}, \emph{Long-parameter list}, \emph{Refused Bequest}, \emph{Speculative Generality}, \emph{Binding Resources too early}, \emph{HashMap usage} и \emph{Private getters and setters}. Для обнаружения этих антипаттернов исследователи использовали инструмент собственной разработки ReCon \cite{morales2017use}, добавив в него поддержку вышеописанных запахов кода. В качестве тестируемого кода были взято несколько проектов из репозитория FDroid\footnote{https://f-droid.org/}, на которых авторы использовали инструмент и вручную исправляли найденные антипаттерны. В результате экспериментов были выделены энергозатратные антипаттерны: \emph{Blob}, \emph{Lazy Class},  \emph{Refused Bequest}, \emph{Binding Resources too early}, \emph{HashMap usage} и \emph{Private getters and setters}.

Fatima et al. \cite{fatima2020tool} провели исследование и сделали обзор существующих инструментов поддержки Android разработки. Найденные инструменты были разделены на 3 категории: профилировщики, детекторы специфических для Android запахов кода и оптимизаторы. В статье был выделен список антипаттернов, которые покрываются найденными инструментами, а также список используемых для их обнаружения техник и типы входных и выходных данных каждого инструмента (например, исходный код приложения или APK). Наиболее типичной техникой обнаружения был статический анализ кода с использованием предопределённого набора правил и запахов кода. В своей следующей работе \cite{fatima2020detection} исследователи представили собственный инструмент обнаружения и исправления запахов кода, основанный на выделенных в предыдущей работе идеях. Новый инструмент является расширением существующего инструмента анализа кода Android Lint. Авторы взяли его за основу, потому что Android Lint имеет интеграцию в официальную среду Android разработки Android Studio, а также имеет API, который позволяет добавлять собственные правила анализа кода. Таким образом исследователи реализовали поддержку 9 новых специфических для Android запахов кода и улучшили поддержку 3 существующих запахов кода.

\begin{table}[]
\begin{tabular}{|l|l|l|l|l|l|}
\hline
\begin{tabular}[c]{@{}l@{}}Подтверждённые \\ антипаттерны\end{tabular} & \begin{tabular}[c]{@{}l@{}}Gottschalk\\ et al.\end{tabular} & \begin{tabular}[c]{@{}l@{}}Cruz\\ et al.\end{tabular} & \begin{tabular}[c]{@{}l@{}}Carette \\ et al.\end{tabular} & \begin{tabular}[c]{@{}l@{}}Palomba \\ et al.\end{tabular} & \begin{tabular}[c]{@{}l@{}}Morales \\ et al.\end{tabular} \\ \hline
Data Transfer                                                          & \checkmark                                    &                                                       &                                                           &                                                           &                                                           \\ \hline
Backlight                                                              & \checkmark                                    &                                                       &                                                           &                                                           &                                                           \\ \hline
Statement Change                                                       & \checkmark                                    &                                                       &                                                           &                                                           &                                                           \\ \hline
\begin{tabular}[c]{@{}l@{}}Third-Party \\ Advertising\end{tabular}     & \checkmark                                    &                                                       &                                                           &                                                           &                                                           \\ \hline
\begin{tabular}[c]{@{}l@{}}Binding Resource \\ Too Early\end{tabular}  & \checkmark                                    &                                                       &                                                           &                                                           & \checkmark                                  \\ \hline
DrawAllocation                                                         &                                                             & \checkmark                              &                                                           &                                                           &                                                           \\ \hline
WakeLock                                                               &                                                             & \checkmark                              &                                                           &                                                           &                                                           \\ \hline
Recycle                                                                &                                                             & \checkmark                              &                                                           &                                                           &                                                           \\ \hline
ObsoleteLayoutParam                                                    &                                                             & \checkmark                              &                                                           &                                                           &                                                           \\ \hline
ViewHolder                                                             &                                                             & \checkmark                              &                                                           &                                                           &                                                           \\ \hline
HashMap Usage                                                          &                                                             &                                                       & \checkmark                                  &                                                           & \checkmark                                  \\ \hline
Internal Getter/Setter                                                 &                                                             &                                                       & \checkmark                                  & \checkmark                                  & \checkmark                                  \\ \hline
\begin{tabular}[c]{@{}l@{}}Member \\ Ignoring \\ Method\end{tabular}   &                                                             &                                                       & \checkmark                                  & \checkmark                                  &                                                           \\ \hline
Leaking Thread                                                         &                                                             &                                                       &                                                           & \checkmark                                  &                                                           \\ \hline
Slow Loop                                                              &                                                             &                                                       &                                                           & \checkmark                                  &                                                           \\ \hline
Blob                                                                   &                                                             &                                                       &                                                           &                                                           & \checkmark                                  \\ \hline
Lazy Class                                                             &                                                             &                                                       &                                                           &                                                           & \checkmark                                  \\ \hline
Refused Bequest                                                        &                                                             &                                                       &                                                           &                                                           & \checkmark                                  \\ \hline
\end{tabular}
\caption{Подтверждённые запахи кода и работы, в которых они исследовались.}
\end{table}

