Мобильные устройства пользовались огромным успехом за последние пять лет, например, в последние 5 лет средние продажи смартфонов составляют более 1,5 млн. в год~\cite{globalsales}, и более 70\% из этих устройств используют операционную систему Android~\cite{worldwide}. По мере увеличения количества устройств количество приложений (или приложений) также быстро росло в последние годы. Следовательно, количество мобильных разработчиков также увеличивается. Приложения в основном пишутся с использованием популярных объектно-ориентиро\-ванных (или объектно-ориентированных) языков программирования, таких как Java, Objective-C, Swift или C\#. 

 Тем не менее, мобильная разработка не так похожа на традиционную разработку программного обеспечения~\cite{wasserman2010software}, и разработчики должны учитывать мобильные особенности. Кроме того, спрос пользователей продолжает расти и вынуждает мобильных разработчиков как можно быстрее добавлять новые функции и поддерживать свои приложения. К сожалению, это давление заставляет разработчиков применять плохие методы реализации, также известные как запахи кода~\cite{fowler2018refactoring}. 
 
 Запахи кода могут привести к утечке ресурсов ЦП, памяти, батареи и т.д.~\cite{cscatalogue}. Утечки могут ухудшить качество приложения с точки зрения стабильности, удобства для пользователя, удобства обслуживания и т.д. Также важно отметить, что более 18\% приложений Android имеют запахи кода~\cite{liu2014characterizing}.
 
 В связи с потенциальным влияением на энергопотребление устройства, а следовательно и времени его работы, необходимо выяснить, какие энергетические запахи кода были изучены. На основе полученных знаний также необходимо разработать инструмент автоматического исправления энергетических запахов кода, который может помочь разработчикам мобильных приложений в создании более энергоэффективного кода.


% Больше инфы можно поискать в слайдах Кознова%~\cite{koznov}.

% Тут  4 части(абзаца) максимум на 2 страницы
% \begin{enumerate}
% \item Background, known information
% \item Knowledge gap, unknown information
% \item  Hypothesis, question, purpose statement 
% \item Approach, plan of attack, proposed solution
% \begin{itemize}
% \item Последний абзац должен читаться и быть понятным в отрыве от остальных трёх
% \end{itemize}
% \end{enumerate}